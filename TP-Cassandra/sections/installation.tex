\par On réalise l'installation du logiciel avec la dernière version en date: \textit{2.1.7}.\newline
\begin{tt}
lmdadm \$ wget ftp://mirrors.ircam.fr/.../apache-cassandra-2.1.7-bin.tar.gz \newline
lmdadm \$ tar -zxvf apache-cassandra-2.1.7-bin.tar.gz \newline
\end{tt}
On redirige en suite les logs en éditant le fichier \textit{logback.xml}. \newline
\begin{tt}
... \newline
\textcolor{cyan}{<file>} /home/lmdadm/log/cassandra/system.log \textcolor{cyan}{</file>} \newline
... \newline
\end{tt}
Puis on édite le fichier \textit{cassandra.yalm}. La liste suivante présente tous la paramètres personnalisés dans le cadre du TP:
\begin{itemize}
\item \textcolor{cyan}{cluster\_name}\textcolor{magenta}{:} CassandraDB
\item \textcolor{cyan}{data\_file\_directories}\textcolor{magenta}{:} /home/lmdadm/data/cassandra
\item \textcolor{cyan}{commitlog\_directory}\textcolor{magenta}{:} /home/lmdadm/log/cassandra/commitlog
\item \textcolor{cyan}{saved\_caches\_directory}\textcolor{magenta}{:} /home/lmdadm/tmp/cassandra/saved\_caches
\end{itemize}
On lance ensuite la base de donnée. On note que l'option indiquée dans le TP -f
signifie: \textit{force foreground}