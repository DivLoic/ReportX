\subsubsection*{02 - 06 Fevrier}
Je suis chargé retrouver l'origine d'un bug dans l'application 360. Cette application fait l'objet de plusieur présentations cette semaine. Le recule pris sur les sources me permet de corriger l'erreur en une demi journée. Le reste de la semain est consacré à l'avancement sur le projet Tracfin.

\subsubsection*{09 - 13 Fevrier}
Je travaille cette semaine sur les scriptes d'intégration de l'application LogSyris. J'automatise l'intégrations des logs, je documente les fonctions et rajoute quelque commentaires. Jusqu'à maintant l'intégration des sources de données était presque manuelle. Même si il n'est pas encore déscidé d'industrialiser le projet je met en place solution intermédiaire pour faciliter le chargement. 

\subsubsection*{16 - 20 Fevrier}
Dévellopement de scripts qui pourrait être utils pour l'intégration des fichiers que l'ont est amené à traiter.

\subsubsection*{23 - 27 Fevrier}
Les deux jours de cette semaine sont consacrés à la livraison du projet EDD (entrepôt de données) dans l'environement d'hologation.

\subsubsection*{02 - 06 Mars}
Livraison en Homologation EDD n1.

\subsubsection*{09 - 13 Mars}
Cette semaine j'assite de nouvelles équipes dans leur installation et prise en main des environements de dévellopement. En effet les prestataires se renouvellent et parmis les nouveaux arrivants se trouvent des consultants ORACLE. Ce qui va certainement nous rapprocher de technologie propriétaires.

\subsubsection*{16 - 20 Mars}
Cette semaine je prépare la mise en homologation n°2 de l'entrepot de données (EDD). Je prépare une documentation pour les opérateurs en prenant en compte les problèmes rencontrés sur la livraison du premier lot. Je retire égallement les parties qui ne sont pas à répéter. La semaine comprend aussi des corrections de bugs sur l'application "parcours utilisateur".

\subsubsection*{23 - 27 Mars}
Livraison en Homologation EDD n2.

\subsubsection*{30 Mars - 03 Avril}
Dans le cadres du projet 360 des overtures de routes sont réalisées. Une erreur à été comise lors de l'ouverture et la mauvaise machine a été traitée. Je dois donc échanger deux applications de machine et les remettre en route

\subsubsection*{06 - 10 Avril}
L'un des pocs de notre entité est décliné en offre de service et un premier client (d'une entité voisine) nous propose un point pour savoir si l'ont peut apporter un valeur intéréssante pour sont application.
J'ai donc les logs de cette application et une journée pour en extraire tous ce que je trouve et le présenter. 

Avec un script python je crée quelques graphiques que je pose sur un slide accompagré de plusieur métrique repérée. Les notions apporté par le cours de System d'exploitation ont été très utils pour cette tâche.

\subsubsection*{13 - 17 Avril}
Je relance l'application Parcours utilisateur qui à cessé de fonctionné en début de semaine. Une dexième réunion à lieu avec un client potentiel à l'offre de service Parcours utilisateur.

\subsubsection*{20 Avril}
Vacances de Pâque.

\subsubsection*{20 - 24 Avril}
La grande mission des vacances est d'utiliser les ouvertures de route établir les flux entre nos petites applications et divers site d'information / réseaux sociaux.

\subsubsection*{04 Mai - 29 Juin}
A la suite d'une réorganisation du service il est déscidé que les projets de type suivit de parcours utilsateurs sont mis de coté. Je me concentre donc sur les projets 360 qui vont repartirent pour de nouveaux cycle de dévellopements dans les environnement SG pour de nouveaux commanditaires.
 

\subsubsection*{Juillet - Août}
\par
	Après la fin des cours, le 29 Juin, débute une période continue en entreprise.
Je suis amené à revenir sur certains points qui sont plus difficiles à suivre à mis-temps.
\begin{itemize}
\item Les décisions de sécurités et d’architecture liées à l’initiative du « Datalake »
\item La nouvelle architecture mise en place dans le cadre de la reprise du POC 360.
\end{itemize}
Dans le cadre du « DataLake » il s’agit essentiellement d’une prise de connaissance sur le socle technologique mis en place. En effet les prochains POC ou projets pourraient cohabiter avec le « DataLake ». La deuxième tache a un aspect bien moins théorique puisqu’il me faut monter en compétence sur les technologies apportées (par les prestataires OCTO) dans le cadre de la version deux du « POC vision 360 ».  La cible pour moi étant de participer activement au développement d’un project.
\par
	Dès le 25 juillet j’agis sans mon supérieur qui est en congé. Durant cette période je
suis principalement chargé d’assurer un suivit sur le POC EDD « Entrepôt de
donnée ». Le projet est relancé par l’unique abonné au service délivré par l’EDD 
(le Dashboard Interactif). Le mois est donc consacré à la correction des bugs constatés 
par nos collaborateurs de l’entité COO, responsable du Dashboard et l’établissement
des différents points liées à l’industrialisation de l’EDD. En dehors de développements,   
J’intervient également sur l’enrichissement des spécification technique et la
documentation. 

\subsubsection*{10 - 25 Septembre}
Dès la rentrée, une nouvelle demande provenant des équipes COO remet en question le travail effectué sur le projet EDD lors des mois Juillet- Août. Il s’agit d’historier des informations transmises aux équipes FAST-IT (en charge du Dashboard interactif RESG). Un collaborateur (FAST-IT) dédié au problème m’aide à comprendre ce qui est attendu. Nous parvenons donc au résultat à la fin du moi. Le premier mois du troisième semestre se partage donc entre cette tâche et la préparation de ma soutenance de mi-parcours.  