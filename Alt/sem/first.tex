\subsubsection*{1er Septembre}
Arrivé et présentation de l'équipe. Présentation des diférents projets et de l’écosystème dans lequel travail la petite équipe constituée de mon manageur et son chef. Six petits projets sont en cours. Ce sont des POC pour « proof of concept », destinés à évoluer pour devenir de vrai sujets d’industrialisation qui viendront enrichir le porte folio du pôle de compétence Big Data de notre entité. 
\subsubsection*{2 Septembre}
J’assiste à la première réunion sur  les POC, « proof of concept ». Les technologies utilisées dans ces projets pourront donc servir à d’autre projets et pour différentes entités.
\subsubsection*{3 Septembre  -  5 Septembre}
La fin de la semaine à été dédiée à la préparation de mon poste de travail, de mon badge, des mots de passes et d’autres sujets administratifs. Je rencontre les consultants externes qui sont sur les projets depuis le lancement de l’initiative Big Data lors des réunions d’avancement pour les POC.
\subsubsection*{8 - 12 Septembre}
Une des finalités des projets big data est la constitution d’un « data lake » pour notre entité. Nous serons donc amenés à nous intéresser au logiciel Hadoop. Dans cette optique j’ai reçu une documentation pour débuter. Les 800 pages de doc ont été fournies à mon manageur à la suite d’une formation. La semaine est également ponctuée par les réunions d’avancement et d’architecture.
\subsubsection*{15 - 19 Septembre}
Cette semaine on m’a présenté les 4 machines dédiées aux POCs en interne. (Ils sont actuellement chez le prestataire CGI). Je m’adapte donc aux machines virtuelles et ce qui est mis à disposition dessus. Je commence l’installation du logiciel hadoop avec la distribution « Apache » sur un seul nœud. La semaine est également ponctuée par les réunions d’avancement et d’architecture.
\subsubsection*{19 Septembre}
Une réunion particulière a également eu lieu. Elle réunit les commanditaires du projet « 360 » (métier, sponsor et maitrise d’ouvrage), notre équipe et les consultants CGI pour la présentation des avancés et des discutions sur les budgets. Elle était particulièrement enrichissante. Elle m’a permis de mieux comprendre dans quelles conditions nous travaillons.
\subsubsection*{22 - 26 Septembre}
Je poursuis l’installation du logiciel Hadoop tout en continuant de progresser dans la documentation. Il y a actuellement des discussions sur l’architecture technique qui concerne en partie la distribution utilisée. Il se peut que je refasse  l’installation avec une autre version. Comme il y a un fort aspect veille technologique dans notre travail il peut être intéressant de comparer les deux. C’est un plus que je peux apporter à l’équipe qui n’a pas forcement le temps de le faire     
\subsubsection*{Jeudi 25 Septembre}
Rentrée scolaire.
\subsubsection*{01 - 10 -Octobre}
Mon Manageur part en formation pendant une semaine sur les sujets Elasticsearch et hadoop. Je suis donc seul et chargé de « tenir la boutique ». Je suis désigné comme point d’entré sur le sujet dont il s’occupe notamment la mise en homologation de l’application KYC. Je fais le lien entre les équipes internes et  les consultants (sopra et CGI). Il a été difficile de tout comprendre et de se faire comprendre car KYC est le seul projet ou je ne suis pas impliqué (ni le développement, ni dans les réunions).
\subsubsection*{Mercredi 1 Octobre}
Premier jours de cours à l’ISEP.
\subsubsection*{06 - Octobre}
Les difficultés rencontrées avec hadoop sont résolues petit à petit. Je prends en note des points bloquants (les dépendances  manquantes) pour permettre à mes collègues d’anticiper les prochaines installations. Le principal problème: Les machines de développement pour des raisons de sécurité sont totalement déconnectées d’internet. La semaine est également ponctuée par les réunions d’avancement et d’architecture.
\subsubsection*{13 - 17 Octobre}
Désormais le lancement de hadoop (la distribution d’Apache)  se passe sans problème. On m’a indiqué de nouvelles consignes de configuration pour respecter l’arborescence normalisée pas l’entreprise. Je retravaille donc cette partie. La semaine est également ponctuée par les réunions d’avancement et d’architecture.
\subsubsection*{17 -  Octobre}
Une réunion très enrichissante à eu lieu aux tours (Défence) avec des participants du secteur bancaire. Il s’agit du point de validation des Use Case pour le lancement du POC TRACFIN. C’est le POC le plus important dans le sens où c’est celui qui fait intervenir la plus grande volumétrie et des compétences de Data scientist tel que le Machine Learning et la Data Discovery. Les aspects « buisiness » et juridique sont fortement présent puisqu’il s’agit de détection de comportements frauduleux.
\subsubsection*{20 - 24 Octobre}
Un passage de connaissance est assuré par les équipes CGI. Nous nous déplaçons donc mon manageur et moi pour assister à une introduction sur les technologies qui composent les petits projets 360. La séance se termine par le passage des sources. Je suis chargé de m’en imprégner. La semaine est également ponctuée par les réunions d’avancement et d’architecture.
\subsubsection*{27 - 31 Octobre}
Vacances  de toussaint \\
Mon manageur part en vacance pour la semaine et je suis de nouveau le point d’entré en cas problème.  Je découvre quelques processus société générale. Je termine mes recherches sur la configuration d’hadoop et recommence avec une nouvelle distribution (Cloudera). Je termine et tien au courant mon manageur de cette avancée par mail.

Je suis également chargé, à l’aide de mes collègues sur place, de monter 3 réunions dédiées à la recherche de nouveaux cas d’utilisations du projet LogSyris. Je réussit à contacter 5 collaborateurs.
\subsubsection*{3 - 7 Novembre}
La séries de trois réunions organisées pendant les vacances débute cette semaine. Elle réunie pour la première fois un nombre étendu de collaborateurs pour une réunion de travail. Je réajuste les ordres de réunions au fur et à mesure et je prépare les comptes rendus. La semaine est également ponctuée par les réunions d’avancement et d’architecture.
\subsubsection*{10 - 14 Novembre}
Mon travail sur hadoop s’arrête ici. Il est déterminé en réunion d’architecture que les outils autour de l’écosystème hadoop seront apportés par une autre entité. Nous nous serviront donc de leur socle technique pour travailler et ne nous soucieront donc pas des installations et configuration. La semaine est également ponctuée par les réunions d’avancement et d’architecture. 
\subsubsection*{17 - 21 Novembre}
Participation lundi et mardi à un évènement interne société générale: Hackathon SG, un concours de développement.
\subsubsection*{24 - 28 Novembre}
Arrivée de nouvelles ressources pour le lancement d’un nouveau projet. Nous recevons également du nouveau matériel qui leur est dédié. Je suis donc chargé de réaliser quelque installations standards et les accompagner sur ce nouvel environnement de travail. Ils m’expliquent ensuite la dimension « Data analyse » qui est un aspect nouveau dans nos projets. La semaine est également ponctuée par les réunions d’avancement et d’architecture.
\subsubsection*{01 - 12 Décembre}
J’assiste les ressources dédiées au projet TRACFIN et essaie de monter en compétence sur hadoop. 
\subsubsection*{15 - 19 Décembre}
Je profite du calme provoqué par les départs en vacances pour prendre de l’avance et rédiger un TP proposé par l’ISEP aux A3, qui portent sur des sujets étudiés à la société générale. La semaine est également ponctuée par les réunions d’avancement et d’architecture.
\subsubsection*{22 - 26 Décembre}
Mon manageur étant en vacance il laisse des objectifs à réaliser pour les deux semaines. Avec l’aide de mes collaborateurs externes je dois réaliser la migration de 4 projets vers les environnements sécurisées de la société général (actuellement sur les serveurs de CGI). Ce qui comprend l’installation des logiciels et le déploiement des sources.
\subsubsection*{25 Décembre }
Joyeux Noel !
\subsubsection*{29  Décembre - 2 Janvier}
Après quelque prises d’initiative la migration d’un des projets prend du retard mais cela m’a permis respecter des process standard et d’établir directement une documentation. Tous est installé à temps et des présentations auront lieu dans les semaines qui suivent.
\subsubsection*{1er Janvier}
Bonne Année.
\subsubsection*{05 - 09 Janvier}
Il m’est demandé de m’imprégner des développements réalisés sur un projet en particulier: LogSyris. Un projet d’analyse de l’utilisation d’une application SG. L’objectif étant de pouvoir le décliner pour une autre application. C’est alors un sujet que je pourrai suivre de bout en bout, autant dans les spécifications que dans les développements. 

La semaine est donc dédiée à l’analyse du code de ce projet. Il fait intervenir des languages/technologies qui me sont inconnus. 
\subsubsection*{12 - 16 Janvier}
Je continue d’analyser et d’assimiler les développements du projet LogSyris. J’en profite pour reformuler et commenter le code qui a été fourni dans un mode POC, c’est à dire en une série de « sprint » très courts.

La semaine est également partagé avec un autre sujet très urgent: la mise en homologation de l’EDD (entrepôt de données). Il m’est donc demandé de rassembler tous ce que j’ai réalisé pendant les vacances de noël pour l’environnement de développement et de rédiger un document de mise en homologation.
\subsubsection*{19 - 23 Janvier}
Des démonstrations sur le projet LogSyris ont lieu. J’intègre de nouvelles données pour qu’il y ai plus de matière à observé. Je participe aux réunions avec la MO de ce projet.

L'homologation de l'EDD prend du retard au niveau de la mise à disposition des machines. J'en profite et je contacte les consultants d’une SS2i (Théodo) employée par une autre entité pour m’aider au sujet de la cohabitation entre leur application et notre EDD. Je propose de faire des essaies dans l’environnement de développement.
\subsubsection*{19 - Janvier}
Mon équipe se réunit pour mettre en place la première offre de service du pôle de compétence de notre entité. Il s’agit de répliquer LogSyris, et de proposer l’analyse de log à d’autres entités. Nous sommes aidés par le manager de l’équipe CGI pour la définition de cette offre de service. Ma première mission en solo sera certainement de répondre à un client SG demendeur de ce type de service. La semaine est également ponctuée par les réunions d’avancement et d’architecture.
\subsubsection*{26 - 30 Janvier}
Je modifie et peaufine le dossier de mise en homologation de l'entrepot de données (EDD). Je devrais, normalement la semaine prochaine, pilloter la livraison et faire le lien avec les oppérationels qui déploient l'application.

La dernière intégration de données se passe très mal pour le projet logSyris. Je prend en note les améliorations que j'aimerais apporter et je les exposes.  
