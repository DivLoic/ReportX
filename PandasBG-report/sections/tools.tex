\par
Pour pouvoir collaborer plus facilement, nous avons déployé un “cluster” Hadoop que nous installons et administrons nous-même. Cela nous a permis d’éviter des problèmes de ressources liés à l’utilisation des nombreux logiciels.
\par
Le cluster est composé de 3 serveurs \textbf{PSV1, PSV2, PSV3} (ces abréviations sont utilisées dans la suite de la description de l'environnement) sur lesquels nous avons installé la distribution Hortonworks de Hadoop (HDP 2.2) et nous permet de travailler ensemble à distance.
\par
Dans un premier temps, il nous faut connecter les serveurs entre eux. Pour cela, on génère avec le premier serveur (\textbf{PSV1}) une clé SSH qui sera copiée sur les deux autres serveurs (dans le fichier knowhost du user root).
\par
Ensuite, on débute l’installation de l’écosystème. On commence donc par l’installation d’Ambari-serveur que l’on a placée sur \textbf{PSV1}.
\par
Par la suite, lorsqu’Ambari serveur est lancé, il nous est demandé de fournir les adresses IP des deux autres serveurs du cluster. Ambari installe alors Ambari-agent sur chaque serveur qui lui se chargera d’installer tous les logiciels Hadoop (il nous propose une répartition des logiciels que l’on a ensuite adaptée à nos ressources)   

\par
\begin{block}{Note sur les agents Ambari} Il consiste en une série de scripts python dédiée à de nombreuses tâches comme l’installation des paquets, la mise en place et l'autorisation des répertoires ou encore le redémarrage des services Hadoop. 
\end{block}
Notre cluster est finalement installé. Nous présenterons donc une répartition des différents logiciels. Chaque machine possède 4GO RAM et possède un Datanode sur un file system de 1 TO. \newline

\begin{figure}[h!]
\centering
    
\begin{tabular}{|c|c|c|c|}
\hline
Note&SV1&SV2&SV3\\ \hline
&Ambari serveur&&\\ \hline
&&namenode&sec namenode\\ \hline
&&Yarn&Yarn\\ \hline
&&&Oozie\\ \hline
&&&Hbase\\ \hline
&&&Hive\\ \hline
&Zookeeper&Zookeeper&Zookeeper\\ \hline
Éteint&&Storm&\\ \hline
Éteint&&Kafka&\\ \hline
&Hue&&\\ \hline
&Livy&&\\ \hline
(client)&Tez&Tez&Tez\\ \hline
(client)&Pig&Pig&Pig\\ \hline
(client)&Sqoop&Sqoop&Sqoop\\ \hline
\end{tabular}

\caption{Répartition des logiciels}
\end{figure}



